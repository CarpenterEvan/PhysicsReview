\usepackage[utf8]{inputenc} % allows for non-ASCII characters
\usepackage[x11names]{xcolor}    % Color extensions
\usepackage[margin=1in]{geometry} % Formatting on page
\usepackage{array} % big arrays
\usepackage{amsmath} % Math symbols
\usepackage{amsthm}   
\usepackage{amssymb}
\usepackage[backend=bibtex]{biblatex} % bibliography
\usepackage{bm} % bold for greek letters
\usepackage{cancel}
\usepackage{enumitem} % Enumerating things
\usepackage{esint} % better double and triple Integrals
\usepackage{etoc}
\usepackage{fancyhdr} % Headers
    \pagestyle{fancy}
\usepackage{float}
\usepackage{gensymb}
\usepackage{physics} % easier commands for physics things
\usepackage{relsize}  % allows for larger/smaller math 
\usepackage{textcomp} % Gets rid of perthousand error
\usepackage{upquote}  % fixes quotes in verbatim environment
\usepackage{verbatim} % Allows for comment environment
\usepackage{tikz} % pictures
\usepackage{pgfplots} % plots and graphics
\usepackage{graphicx} % plots and graphics
    \pgfplotsset{compat=newest} % or use compat=1.6
    \usetikzlibrary{calc}
    \usetikzlibrary{3d}
\usepackage{xparse} % Better commands
\usepackage[hidelinks]{hyperref}% References--THIS GOES LAST

% Packages that this breaks without: amsmath, gensymb, physics, hyperref, xcolor, xparse, and possibly others
\numberwithin{equation}{section} % amsmath command that renews equation counter in each section\
\def\ints{\int_\mathcal{S}} % Surface integral
\def\intv{\int_\mathcal{V}} % Volume integral with V subscript
\def\intall{\int_{-\infty}^{\infty}} % Integral over all space
\def\iintall{\iint_{\rm All Space}} % Integral over all space
\def\iiintall{\iiint_{\rm All Space}} % Integral over all space
\def\ik{4\pi\epsilon_0} % Inverse k for EM
\def\lap{\mathcal{L}}
\def\answerline{ % double horizontal line placed 0.5 cm below text, space between lines is 0.07 cm, then 0.75 cm of space 
	\vspace{0.5 cm}
	\hrule
	\vspace{0.07 cm}
	\hrule
	\vspace{0.75 cm}\noindent} % don't indent text after the line
\NewDocumentCommand\length{O{3pt}}{\setlength\jot{#1}} % for align vertical spacing, there's probably a better way to do this locally
\NewDocumentCommand\ft{s O{n} O{L}}{ % I dont want to write \sin(stuff) \cos(stuff) every time in fourier transforms (ft)
    \IfBooleanTF{#1}{
        \sin(\frac{#2 \pi}{#3}x)
    }{
        \cos(\frac{#2 \pi}{#3}x)
    }
}

\NewDocumentCommand\dl{s}{\IfBooleanTF{#1}{}{\cdot} d\mathbf{l}} % quicker curve integral dl, if no star, it also makes the dot 
\NewDocumentCommand\da{s}{\IfBooleanTF{#1}{}{\cdot} d\mathbf{a}} % quicker surface integral da, if no star, it also makes the dot 
\NewDocumentCommand\oo{O{1} m} {\frac{#1}{#2}}   % reciprocal- 'one over', option to make it a regular fraction because oo is still quicker to type.
\NewDocumentCommand\thetitle{m O{}}{ \title{ \hypertarget{top}{\textbf{#1}} \\ \large {#2} } } %Bold title that is a hypertarget, optional bolded subtitle that's scaled properly 

\NewDocumentCommand \e {s m}{ % basis vector command
	\IfBooleanTF{#1} % \e{x} for x hat notation, \e*{x} for e_x notation
	{\mathbf{\hat{e}_{#2}}}
	{\bm{\hat{#2}}}} % from package bm, more powerful than \mathbf} 
\NewDocumentCommand \parametric {m}{ %command for getting boundary conditions with a "{" on the left
    \left\{ 
	\begin{gathered}
		\begin{matrix}
			#1
    		\end{matrix}
	\end{gathered}
	 \right.}
\NewDocumentCommand \der {s O{} m g}{ % custom derivative command
     \IfBooleanTF{#1} % if \der*, #1 is true, euler notation used, if \der , #1 is false, d/dx is used 
    {\IfNoValueTF{#4}  % g returns -NoValue- if no argument (read below)
        {\mathrm{D}_{#3}^{#2}}  % g is included so \der{x} and \der{f}{x} both have x in the denominator
        {\mathrm{D}_{#4}^{#2}#3}
            }
    {\IfNoValueTF{#4}
        {\frac{\mathrm{d}^{#2}}{\mathrm{d} #3^{#2}}}
        {\frac{\mathrm{d}^{#2} #3}{\mathrm{d} #4^{#2}}}
	}}    
\NewDocumentCommand \pder {s O{} m g}{ % custom partial derivative command, allows for euler notation if starred
     \IfBooleanTF{#1}
        {\IfNoValueTF{#4}
            {\partial_{#3}^{#2}}
            {\partial_{#4}^{#2}#3}
                }
        {\IfNoValueTF{#4}
            {\frac{\partial^{#2}}{\partial #3^{#2}}}
            {\frac{\partial^{#2} #3}{\partial #4^{#2}}}
                }} 
                
\NewDocumentCommand \header {m m m}{
	\fancyhead[L]{#1} 
	\fancyhead[C]{\hyperlink{top}{ \textbf{#2} }}
	\fancyhead[R]{#3}
	\pagestyle{fancy}
	\setlength\headheight{17pt}}
    
\NewDocumentCommand{\coloredanswer}{O{LavenderBlush2} m}{ % makes coloredbox with pretty color
	\mathchoice
	{\colorbox{#1}{$\displaystyle #2$}}
	{\colorbox{#1}{$\textstyle #2$}}
	{}
	{}}

\newcounter{probcount} % new counter for problem numbers, starts at 0 by default
\NewDocumentEnvironment{ problem } {O{} +b} %Problem environment for homework, autocounts numbers, behaves like a section and can get listed (and hyperlinked) in the toc. 
	{\addtocounter{probcount}{1} % increase counter at the beginning of every problem
	 \phantomsection % invisible page marker for hyperref
	 \addcontentsline{toc}{subsection}{Problem \theprobcount. {\it#1}} % add "Problem \theprobcount. \textit{#1}" to table of contents (toc) as if it were a subsection
	 \begin{trivlist} % begin unmarked ("trivial") list
	 \item {\bf Problem \theprobcount.} {\it#1} % print Problem with the counter and optional text 
	 \item #2  % +b is for inputs that are long
	 \end{trivlist} % end list
	}{} % don't do anything then "problem" environment ends, irrelevant because trivlist is already ended
	
\def\UM{\colorbox{blue}{\textcolor{Gold1}{\textbf{University of Michigan}}}} % School Pride
\def\EC{\href{https://github.com/CarpenterEvan/PhysicsReview}{\(\mathbb{E}\textsc{van}~\mathbb{C}\textsc{arpenter}\)}}
\endinput