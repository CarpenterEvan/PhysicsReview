% Since \input doesn't like preambles, only uncomment next line if you're editing this TeX, don't forget the \end{document}! 
%\documentclass{article}\usepackage[utf8]{inputenc} % allows for non-ASCII input characters like ä
\usepackage[x11names]{xcolor}% Color extensions, load first
\usepackage[margin=1in]{geometry} % Formatting on page
\usepackage{array} % better arrays
\usepackage{amsmath} % Math symbols and formats
    \numberwithin{equation}{section} % amsmath command that renews equation counter in each section\
\usepackage{amsthm}   
\usepackage{amssymb}
\usepackage[backend=bibtex]{biblatex} % bibliography
\usepackage{bm} % bold, even for greek letters
\usepackage{enumitem} % Enumerating things
\usepackage{esint} % better double and triple Integrals
\usepackage{etoc} % super powerful table of contents
\usepackage{fancyhdr} % Headers
    \pagestyle{fancy}
\usepackage{gensymb} % more symbols
\usepackage{physics} % easier commands for physics things
\usepackage{relsize}  % allows for larger/smaller math 
\usepackage{textcomp} % Gets rid of perthousand error
\usepackage{upquote}  % fixes quotes in verbatim environment
\usepackage{verbatim} % Allows for comment environment
\usepackage{tikz} % pictures and drawings
    \usetikzlibrary{calc}
    \usetikzlibrary{3d}
    %\usetikzlibrary{external}
    %\tikzexternalize
    %\tikzsetexternalprefix{figures/}
\usepackage{pgfplots} % plots and graphics
    \pgfplotsset{compat=newest} % or use compat=1.6 
\usepackage{graphicx} % plots and graphics
\usepackage{xparse} % Better commands
\usepackage[hidelinks]{hyperref}% References--THIS GOES LAST
\usepackage[utf8]{inputenc} % allows for non-ASCII characters
\usepackage[x11names]{xcolor}    % Color extensions
\usepackage[margin=1in]{geometry} % Formatting on page
\usepackage{array} % big arrays
\usepackage{amsmath} % Math symbols
\usepackage{amsthm}   
\usepackage{amssymb}
\usepackage[backend=bibtex]{biblatex} % bibliography
\usepackage{bm} % bold for greek letters
\usepackage{cancel}
\usepackage[format=hang]{caption}
\usepackage{enumitem} % Enumerating things
\usepackage{esint} % better double and triple Integrals
\usepackage{etoc}
\usepackage{fancyhdr} % Headers
    \pagestyle{fancy}
\usepackage{float}
\usepackage{gensymb}
\usepackage{physics} % easier commands for physics things
\usepackage{relsize}  % allows for larger/smaller math 
\usepackage{textcomp} % Gets rid of perthousand error
\usepackage{upquote}  % fixes quotes in verbatim environment
\usepackage{verbatim} % Allows for comment environment
\usepackage{tikz} % pictures
	\usetikzlibrary{calc}
	\usetikzlibrary{decorations.markings}
	\usetikzlibrary{3d}
	\usetikzlibrary{intersections}
\usepackage{pgfplots} % plots and graphics
	\pgfplotsset{compat=newest} % or use compat=1.6
	\usepgfplotslibrary{fillbetween}
\usepackage{graphicx} % plots and graphics
\usepackage{wrapfig}
\usepackage{xparse} % Better commands
\usepackage[hidelinks]{hyperref}% References--THIS GOES LAST

% Packages that this breaks without: amsmath, gensymb, physics, hyperref, xcolor, xparse, and possibly others
\numberwithin{equation}{section} % amsmath command that renews equation counter in each section\
\def\ints{\int_\mathcal{S}} % Surface integral
\def\intv{\int_\mathcal{V}} % Volume integral with V subscript
\def\intall{\int_{-\infty}^{\infty}} % Integral over all space
\def\iintall{\iint_{\rm All Space}} % Integral over all space
\def\iiintall{\iiint_{\rm All Space}} % Integral over all space
\def\ik{4\pi\epsilon_0} % Inverse k for EM
\def\lap{\mathcal{L}}
\def\answerline{ % double horizontal line placed 0.5 cm below text, space between lines is 0.07 cm, then 0.75 cm of space 
	\vspace{0.5 cm}
	\hrule
	\vspace{0.07 cm}
	\hrule
	\vspace{0.75 cm}\noindent} % don't indent text after the line
\NewDocumentCommand\length{O{3pt}}{\setlength\jot{#1}} % for align vertical spacing, there's probably a better way to do this locally
\NewDocumentCommand\ft{s O{n} O{L}}{ % I dont want to write \sin(stuff) \cos(stuff) every time in fourier transforms (ft)
    \IfBooleanTF{#1}{
        \sin(\frac{#2 \pi}{#3}x)
    }{
        \cos(\frac{#2 \pi}{#3}x)
    }
}

\NewDocumentCommand\dl{s}{\IfBooleanTF{#1}{}{\cdot} d\mathbf{l}} % quicker curve integral dl, if no star, it also makes the dot 
\NewDocumentCommand\da{s}{\IfBooleanTF{#1}{}{\cdot} d\mathbf{a}} % quicker surface integral da, if no star, it also makes the dot 
\NewDocumentCommand\oo{O{1} m} {\frac{#1}{#2}}   % reciprocal- 'one over', option to make it a regular fraction because oo is still quicker to type.
\NewDocumentCommand\thetitle{m O{}}{ \title{ \hypertarget{top}{\textbf{#1}} \\ \large {#2} } } %Bold title that is a hypertarget, optional bolded subtitle that's scaled properly 

\NewDocumentCommand \e {s m}{ % basis vector command
	\IfBooleanTF{#1} % \e{x} for x hat notation, \e*{x} for e_x notation
	{\mathbf{\hat{e}_{#2}}}
	{\bm{\hat{#2}}}} % from package bm, more powerful than \mathbf} 
\NewDocumentCommand \parametric {m}{ %command for getting boundary conditions with a "{" on the left
    \left\{ 
	\begin{gathered}
		\begin{matrix}
			#1
    		\end{matrix}
	\end{gathered}
	 \right.}
\NewDocumentCommand \der {s O{} m g}{ % custom derivative command
     \IfBooleanTF{#1} % if \der*, #1 is true, euler notation used, if \der , #1 is false, d/dx is used 
    {\IfNoValueTF{#4}  % g returns -NoValue- if no argument (read below)
        {\mathrm{D}_{#3}^{#2}}  % g is included so \der{x} and \der{f}{x} both have x in the denominator
        {\mathrm{D}_{#4}^{#2}#3}
            }
    {\IfNoValueTF{#4}
        {\frac{\mathrm{d}^{#2}}{\mathrm{d} #3^{#2}}}
        {\frac{\mathrm{d}^{#2} #3}{\mathrm{d} #4^{#2}}}
	}}    
\NewDocumentCommand \pder {s O{} m g d[]}{ % custom partial derivative command, allows for euler notation if starred, added d[] entry for parenthesis around derivative for quantities that are held constant 
	\IfBooleanTF{#1} % Star for euler notation
	   {\IfNoValueTF{#4} % Logic with #4 is so first argument gets put in denomenator if there is only one ( like \pder{x} ), but if there are two arguments (#4 is second argument) then the second argument is put in denominator \pder{f}{x}
		   {\partial_{#3}^{#2}}
		   {\partial_{#4}^{#2}#3}
			   }
	   {\IfNoValueTF{#5} % #5 is what is held constant
		   {\IfNoValueTF{#4}
			   {\frac{\partial^{#2}}{\partial #3^{#2}}}
			   {\frac{\partial^{#2} #3}{\partial #4^{#2}}}}
		   {{\IfNoValueTF{#4} 
			   {\left(\frac{\partial^{#2}}{\partial #3^{#2}}\right)_{\!#5}\!\!} % \! is thin negative space
			   {\left(\frac{\partial^{#2} #3}{\partial #4^{#2}}\right)_{\!#5}\!\!}}}
			   }} 
                
\NewDocumentCommand \header {m m m}{
	\fancyhead[L]{#1} 
	\fancyhead[C]{\hyperlink{top}{ \textbf{#2} }}
	\fancyhead[R]{#3}
	\fancyfoot[C]{--\thepage--}
	\pagestyle{fancy}
	\setlength\headheight{17pt}}
    
\NewDocumentCommand{\coloredanswer}{O{LavenderBlush2} m}{ % makes coloredbox with pretty color
	\mathchoice
	{\colorbox{#1}{$\displaystyle #2$}}
	{\colorbox{#1}{$\textstyle #2$}}
	{}
	{}}

\newcounter{probcount} % new counter for problem numbers, starts at 0 by default
\NewDocumentEnvironment{ problem } {O{} +b} %Problem environment for homework, autocounts numbers, behaves like a section and can get listed (and hyperlinked) in the toc. 
	{\addtocounter{probcount}{1} % increase counter at the beginning of every problem
	 \phantomsection % invisible page marker for hyperref
	 \addcontentsline{toc}{subsection}{Problem \theprobcount~{\it#1}} % add "Problem \theprobcount. \textit{#1}" to table of contents (toc) as if it were a subsection
	 \begin{trivlist} % begin unmarked ("trivial") list
	 \item {\bf Problem \theprobcount} {\it#1} % print Problem with the counter and optional text 
	 \item #2  % +b is for inputs that are long
	 \end{trivlist} % end list
	}{} % don't do anything then "problem" environment ends, irrelevant because trivlist is already ended
	
\def\UM{\colorbox{blue}{\textcolor{Gold1}{\textbf{University of Michigan}}}} % School Pride
\def\EC{\href{https://github.com/CarpenterEvan/PhysicsReview}{\(\mathbb{E}\textsc{van}~\mathbb{C}\textsc{arpenter}\)}}
\endinput\begin{document}
\section{Differential Equations} 
\invisiblelocaltableofcontents\label{toc:diffeq}
Differential equations are equalities made from functions and their derivatives. Ordinary differential equations (ODEs) only have 1-D derivatives, while partial differential equations (PDEs) have partial derivatives in multiple dimensions. The order of a differential equation is the determined by the highest order derivative present in the differential equation. Since this is physics oriented, the highest order I'll explicitly cover right now is second order. 
\par
\begingroup
    \parindent 0cm \parfillskip 0cm \leftskip 0cm \rightskip 0cm
    \etocframedstyle[1]{\Large\bf Table of Contents}
    \etocsetstyle{subsection}
                  {}
                  {\leavevmode\leftskip 0cm\relax} {\normalsize\bf\etocnumber~\rm\etocname\nobreak\leaders\hbox{.}\hfill\tt p.\makebox[12 pt][r]{\etocpage} \par}
                  {\vspace{-0.25cm}}
    \etocsetstyle{subsubsection}
                  {}
                  {\leavevmode\leftskip 1cm\relax}
                  {\small\bf\etocnumber~\rm\etocname\par}
                  {}
    \tableofcontents\ref{toc:diffeq}
\endgroup\newpage
\subsection{First Order ODEs}
    \subsubsection{Homogeneous}
        A function of the form $f(x,y,y'...)=0$. If the equation as a whole is linear\footnote{An example of a nonlinear differential equation could be $f(x,y,y')=\sin(y'(x))+y^2(x)=0$}, 
        the solution will be of the form \(y=e^{mx}\) where m is some complex number ($m=a+bi$). Solutions start by plugging in this guess of $y=e^{mx}$ and finding $m$:
        {
        \length[0.25 cm]
        \begin{gather*}
            ay''+by'+cy=0 
            \\
            a\qty(m^2e^{m x})+b \qty(m e^{m x})+c \qty(e^{m x})=0
            \\
            \qty(am^2+b m+c)e^{mx}=0
            \\
            am^2+b m+c=0
            \\
            m=\frac{-b\pm\sqrt{b^2-4ac}}{2a}
        \end{gather*}
        }
        Solve for $m$, since $m$ is a complex root there are three cases:
        \begin{gather}
            \underset { a_1 \neq a_2\And b=0 }{\textbf{Real Distinct Roots:}}\quad y_c=c_1e^{a_1x}+c_2e^{a_2x}\\[0.7 cm]
            \underset{a_1=a_2 \And b=0 }{\textbf{Real Repeating Roots:}}\quad y_c=c_1e^{ax}+c_2xe^{ax}\\[0.7 cm]
            \underset{b \neq 0}{\textbf{Complex Roots:}}\quad y_c=e^{ax}\left(c_1\cos(bx)+c_2\sin(bx)\right)
        \end{gather}
    \subsubsection{Non-Homogeneous}
        Takes the form: $y'+P(x)y=g(x)$. 
        \begin{align*}
            \mu(x)
            &\equiv e^{\int P(x) dx}\\
            \frac{d}{dx}(\mu(x)y)=\mu(x)g(x) 
            &\xrightarrow{Move\ dx,\ integrate} \mu(x)y +c_1=\int \mu(x) g(x)dx\\
            y(x)
            &=\frac{\int \mu(x) g(x)dx - c_1}{\mu(x)}
        \end{align*}

        \subsubsection*{Variation of Parameters}
        Not limited by non-constant coefficients. 
        \begin{align*}
            y_p=y_c\int{\frac{g(x)}{y_c}dx}
        \end{align*}    
        \subsubsection{Reduction of Order}
\subsection{Second Order ODEs}
    \subsubsection{Homogeneous}
    \subsubsection{Non-Homogeneous}
        Takes the form: $y''+Q(x)y'+P(x)y=g(x)$
    \subsubsection{Variation of Parameters}
        The general idea is to replace the coefficients ($c_1,c_2$) with functions. 
        $$y_c=c_1y_1(x)+c_2y_2(x)$$
        $$y_p=u_1(x)y_1(x)+u_2(x)y_2(x)$$
        For the case of a second order, the relations of $u(x)$ are depicted below
        \begin{align*}
            u_1'&=\frac{W_1}{W} & u_2'&=\frac{W_2}{W}\\[0.7 cm]
            W&=
            \begin{vmatrix}
            y_1 & y_2 \\
            y_1'& y_2'
            \end{vmatrix}
            &
            W_1&=
            \begin{vmatrix}
            0 & y_2 \\
            g(x)& y_2'
            \end{vmatrix}
            &
            W_2&=
            \begin{vmatrix}
            y_1 & 0 \\
            y_1'& g(x)
            \end{vmatrix}
        \end{align*}
        In summary: Find Wronskian (determinant represented by W). Find $u_i'$. Integrate to get $u_i$ and plug in to $y_p$, then add to $y_c$.

    \subsubsection{Undetermined Coefficients}
        There are two subcategories of this method: The superposition approach and the annihilator approach.\\
        Superposition: Solve for the complementary function $y_c$ (shown in the 'Homogeneous' section), then find a particular solution $y_p$.
\subsection{System of Linear Equations}
    Multiple differential equations that are related to each other. They are typically solved by putting them into matrices and using eigenvalues/eigenvectors.
    \begin{align*}
        \textbf{First Order:}%{
        \begin{bmatrix}
        x'\\
        y'
        \end{bmatrix}
        =
        \begin{bmatrix}
        a&b\\
        c&d\\
        \end{bmatrix}
        \begin{bmatrix}
        x\\
        y
        \end{bmatrix}\implies \textbf{X'}=M\textbf{X}\\
        \textbf{Solving:}\; |M-\lambda I|=0\quad\big{|}\quad  (M-\lambda_iI)\vb{e}_i=0 \quad\big{|}\quad \textbf{X}=\sum_i\vec{c}_ie^{\lambda_ix}\\%}
        \textbf{Normal Modes:}%{
        \begin{bmatrix}
        x''\\
        y''
        \end{bmatrix}
        =
        \begin{bmatrix}
        a&b\\
        c&d\\
        \end{bmatrix}
        \begin{bmatrix}
        x\\
        y
        \end{bmatrix}\implies \textbf{X''}=M\textbf{X}\\
        \textbf{Solving:}\; |M-\omega^2 I|=0 %}
    \end{align*}
\newpage


\subsection{The Laplace Transform}
    A powerful method for solving IVPs using an integral transform. The general method is to transform a differential equation from the $t$ domain to the $s$ domain using the transform, where the equation becomes a simple algebra system. After solving for $Y(s)$, use the inverse transform to turn the obtained function into the complete solution.  Using Partial Fraction Decomposition is often useful when solving these. Use tables.
    \begin{align*}
        \lap[f'](s)&=s\lap[f](s)-f(0)
        \\
        \lap[f''](s)&=s^2\lap[f](s)-sf(0)-f'(0)
        \\
        \lap[f'''](s)&=s^3\lap[f](s)-s^2f(0)-sf'(0)-f''(0)
        \\
        \vdots
    \end{align*}
    Laplace Convolution of two functions $f,g$ is defined to be 
    \[
    (f*g)(t)=\int_0^tf(\tau)g(t-\tau)d\tau
    \]
    If $\lap[f](s)=F(s)~\&~\lap[g](s)=G(s)$ exists, then $\lap^{-1}[FG]=(f*g)$ and $\lap[f*g](s)=FG$. This is useful for when we want to recover $h(t)$ from $H(s)=FG$ for a known $FG$.
\subsection{Partial Fraction Decomposition}
Useful for re-writing some of the results of a Laplace transform. It involves decomposing the denominator of some difficult fraction into multiple seperate fractions. 
\begin{align*}
    \frac{1}{x^2-x+1}=\frac{A}{}
\end{align*}
\newpage
\subsection{Bernoulli Equations: \protect\(y'+P(x)y=g(x)y^n\protect\protect\)}
    Take the form: $y'+P(x)y=g(x)y^n$ for $n \in \mathbb{R}$ When $n \neq 0,1$ solve by substituting $u=y^{1-n}$
    \begin{align*}
        y'+\frac{1}{x}y&=xy^2\xrightarrow{n=2\ \therefore\ u=y^{-1}\ \therefore\ y=u^{-1}} \frac{dy}{dx}=\frac{dy}{du}\frac{du}{dx}=-u^{-2}\frac{du}{dx}\\
        -u^{-2}\frac{du}{dx}+\frac{1}{ux}&=xu^{-2}\xrightarrow{rearrange}\frac{du}{dx}-\frac{1}{x}u=-x\\
        \mu(x)&=e^{-\int 1/x\ dx}=e^{-\ln(x)}=\frac{1}{x}\\
        \int d(\mu(x)u)&=\int \mu(x)g(x)dx\ ;\ \int d\left(\frac{u}{x}\right)=\int (-1)dx\\
        u&=-x^2+c_1x\ \therefore\ \boxed{y=\frac{1}{-x^2+c_1x}}
        \end{align*}
\subsection*{Cauchy-Euler: $x^ny^{(n)}+...+x^2y''+axy'+by=0$}
    Also called The ``Equidimensional'' Equation. Assume solution of $y=x^m$ and plug in. Solve for $m$.
    \begin{alignat}{2}
        \underset { m_1 \neq m_2}{\textbf{Real Distinct Roots:}}
        &\quad y=c_1x^{m_1}+c_1x^{m_2}\\[0.7 cm]
        \underset{m_1=m_2}{\textbf{Real Repeating Roots:}}
        &\quad y=c_1\ln(x)x^m+c_2x^m\\[0.7 cm]
        \underset{b \neq 0}{\textbf{Complex Roots:}}
        &\quad y=c_{1}x^{\alpha }\cos(\beta \ln(x))+c_{2}x^{\alpha }\sin(\beta \ln(x))
    \end{alignat}
\subsection{Numerical}

    What types of numerical differential equations do you do again\ldots?
\subsection{PDEs in Physics}
    \subsubsection{The Heat Equation} % most general and applications 
    \subsubsection{The Wave Equation} % 1d, 3d, spherical, cylindrical
    \subsubsection{Laplace's Equation} % in multi dimensions and coords
    \subsubsection{Poisson's Equation} % same
%\end{document}