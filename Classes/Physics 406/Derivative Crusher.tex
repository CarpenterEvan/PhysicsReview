\documentclass{article}
    \usepackage[utf8]{inputenc} % allows for non-ASCII characters
\usepackage[x11names]{xcolor}    % Color extensions
\usepackage[margin=1in]{geometry} % Formatting on page
\usepackage{array} % big arrays
\usepackage{amsmath} % Math symbols
\usepackage{amsthm}   
\usepackage{amssymb}
\usepackage[backend=bibtex]{biblatex} % bibliography
\usepackage{bm} % bold for greek letters
\usepackage{cancel}
\usepackage[format=hang]{caption}
\usepackage{enumitem} % Enumerating things
\usepackage{esint} % better double and triple Integrals
\usepackage{etoc}
\usepackage{fancyhdr} % Headers
    \pagestyle{fancy}
\usepackage{float}
\usepackage{gensymb}
\usepackage{physics} % easier commands for physics things
\usepackage{relsize}  % allows for larger/smaller math 
\usepackage{textcomp} % Gets rid of perthousand error
\usepackage{upquote}  % fixes quotes in verbatim environment
\usepackage{verbatim} % Allows for comment environment
\usepackage{tikz} % pictures
	\usetikzlibrary{calc}
	\usetikzlibrary{decorations.markings}
	\usetikzlibrary{3d}
	\usetikzlibrary{intersections}
\usepackage{pgfplots} % plots and graphics
	\pgfplotsset{compat=newest} % or use compat=1.6
	\usepgfplotslibrary{fillbetween}
\usepackage{graphicx} % plots and graphics
\usepackage{wrapfig}
\usepackage{xparse} % Better commands
\usepackage[hidelinks]{hyperref}% References--THIS GOES LAST

% Packages that this breaks without: amsmath, gensymb, physics, hyperref, xcolor, xparse, and possibly others
\numberwithin{equation}{section} % amsmath command that renews equation counter in each section\
\def\ints{\int_\mathcal{S}} % Surface integral
\def\intv{\int_\mathcal{V}} % Volume integral with V subscript
\def\intall{\int_{-\infty}^{\infty}} % Integral over all space
\def\iintall{\iint_{\rm All Space}} % Integral over all space
\def\iiintall{\iiint_{\rm All Space}} % Integral over all space
\def\ik{4\pi\epsilon_0} % Inverse k for EM
\def\lap{\mathcal{L}}
\def\answerline{ % double horizontal line placed 0.5 cm below text, space between lines is 0.07 cm, then 0.75 cm of space 
	\vspace{0.5 cm}
	\hrule
	\vspace{0.07 cm}
	\hrule
	\vspace{0.75 cm}\noindent} % don't indent text after the line
\NewDocumentCommand\length{O{3pt}}{\setlength\jot{#1}} % for align vertical spacing, there's probably a better way to do this locally
\NewDocumentCommand\ft{s O{n} O{L}}{ % I dont want to write \sin(stuff) \cos(stuff) every time in fourier transforms (ft)
    \IfBooleanTF{#1}{
        \sin(\frac{#2 \pi}{#3}x)
    }{
        \cos(\frac{#2 \pi}{#3}x)
    }
}

\NewDocumentCommand\dl{s}{\IfBooleanTF{#1}{}{\cdot} d\mathbf{l}} % quicker curve integral dl, if no star, it also makes the dot 
\NewDocumentCommand\da{s}{\IfBooleanTF{#1}{}{\cdot} d\mathbf{a}} % quicker surface integral da, if no star, it also makes the dot 
\NewDocumentCommand\oo{O{1} m} {\frac{#1}{#2}}   % reciprocal- 'one over', option to make it a regular fraction because oo is still quicker to type.
\NewDocumentCommand\thetitle{m O{}}{ \title{ \hypertarget{top}{\textbf{#1}} \\ \large {#2} } } %Bold title that is a hypertarget, optional bolded subtitle that's scaled properly 

\NewDocumentCommand \e {s m}{ % basis vector command
	\IfBooleanTF{#1} % \e{x} for x hat notation, \e*{x} for e_x notation
	{\mathbf{\hat{e}_{#2}}}
	{\bm{\hat{#2}}}} % from package bm, more powerful than \mathbf} 
\NewDocumentCommand \parametric {m}{ %command for getting boundary conditions with a "{" on the left
    \left\{ 
	\begin{gathered}
		\begin{matrix}
			#1
    		\end{matrix}
	\end{gathered}
	 \right.}
\NewDocumentCommand \der {s O{} m g}{ % custom derivative command
     \IfBooleanTF{#1} % if \der*, #1 is true, euler notation used, if \der , #1 is false, d/dx is used 
    {\IfNoValueTF{#4}  % g returns -NoValue- if no argument (read below)
        {\mathrm{D}_{#3}^{#2}}  % g is included so \der{x} and \der{f}{x} both have x in the denominator
        {\mathrm{D}_{#4}^{#2}#3}
            }
    {\IfNoValueTF{#4}
        {\frac{\mathrm{d}^{#2}}{\mathrm{d} #3^{#2}}}
        {\frac{\mathrm{d}^{#2} #3}{\mathrm{d} #4^{#2}}}
	}}    
\NewDocumentCommand \pder {s O{} m g}{ % custom partial derivative command, allows for euler notation if starred
     \IfBooleanTF{#1}
        {\IfNoValueTF{#4}
            {\partial_{#3}^{#2}}
            {\partial_{#4}^{#2}#3}
                }
        {\IfNoValueTF{#4}
            {\frac{\partial^{#2}}{\partial #3^{#2}}}
            {\frac{\partial^{#2} #3}{\partial #4^{#2}}}
                }} 
                
\NewDocumentCommand \header {m m m}{
	\fancyhead[L]{#1} 
	\fancyhead[C]{\hyperlink{top}{ \textbf{#2} }}
	\fancyhead[R]{#3}
	\fancyfoot[C]{--\thepage--}
	\pagestyle{fancy}
	\setlength\headheight{17pt}}
    
\NewDocumentCommand{\coloredanswer}{O{LavenderBlush2} m}{ % makes coloredbox with pretty color
	\mathchoice
	{\colorbox{#1}{$\displaystyle #2$}}
	{\colorbox{#1}{$\textstyle #2$}}
	{}
	{}}

\newcounter{probcount} % new counter for problem numbers, starts at 0 by default
\NewDocumentEnvironment{ problem } {O{} +b} %Problem environment for homework, autocounts numbers, behaves like a section and can get listed (and hyperlinked) in the toc. 
	{\addtocounter{probcount}{1} % increase counter at the beginning of every problem
	 \phantomsection % invisible page marker for hyperref
	 \addcontentsline{toc}{subsection}{Problem \theprobcount. {\it#1}} % add "Problem \theprobcount. \textit{#1}" to table of contents (toc) as if it were a subsection
	 \begin{trivlist} % begin unmarked ("trivial") list
	 \item {\bf Problem \theprobcount.} {\it#1} % print Problem with the counter and optional text 
	 \item #2  % +b is for inputs that are long
	 \end{trivlist} % end list
	}{} % don't do anything then "problem" environment ends, irrelevant because trivlist is already ended
	
\def\UM{\colorbox{blue}{\textcolor{Gold1}{\textbf{University of Michigan}}}} % School Pride
\def\EC{\href{https://github.com/CarpenterEvan/PhysicsReview}{\(\mathbb{E}\textsc{van}~\mathbb{C}\textsc{arpenter}\)}}
\endinput

\header{Physics 406}{}{Derivative Crusher Algorithm}
\thetitle{Derivative Crusher Algorithm}[by\large\ \Akhoury]
\date{Winter 2022}

\author{\small Typed in \LaTeX\ by \EC}
\addtolength{\jot}{0.25cm}% changes row spacing to 0.25cm in align
\def\Akhoury{
    \hyperlink{
        https://lsa.umich.edu/physics/people/faculty/akhoury.html
    }{Ratindranath Akhoury}
    }

\begin{document}
\maketitle\thispagestyle{fancy}
\noindent\textbf{Idea:} A partial derivative with thermodynamically significant variables
    \begin{center}
        $\displaystyle\pder{x}{y}[z]$ where $\{x,y,z\}$ can be any of $\{S,T,P,V,E,F,G,H\}$
    \end{center}
    can be expressed in terms of first and second derivatives of fundamental relations. This re-arranging of expressions is very useful in thermodynamics and statistical mechanics, as it allows us to find simple relations between physical quantities in a thermodynamical system. The 
    Derivative Crusher Algorithm is an algorithm that "crushes" a thermodynamically significant partial derivative into useful and easily measureable quantities. Developed and taught by Professor \Akhoury.
\section*{Choosing Gibbs Free Energy}
    \begin{table}[H]
        \renewcommand{\arraystretch}{1.5}
        \centering
        \begin{tabular}{|c|m{5.5cm}|m{5cm}|}
            \hline
            Variable & Easy to keep fixed? & Easy to measure change? 
            \\\hline
            $S$ & \textcolor{SpringGreen3}{\textbf{Yes}}, Adiabatic processes & \textcolor{OrangeRed3}{\textbf{No}}, change in S depends on change in Q, which is hard to measure
            \\\hline
            $T$ & \textcolor{SpringGreen3}{\textbf{Yes}}, with a heat bath or feedback system& \textcolor{SpringGreen3}{\textbf{Yes}}, by using a thermometer
            \\\hline
            $V$ & \textcolor{OrangeRed3}{\textbf{No}}, not easily, thermal expansion of the container is a byproduct of the container being thermally conductive, the volume changes with heat & \textcolor{SpringGreen3}{\textbf{Yes}}
            \\\hline
            $P$ & \textcolor{SpringGreen3}{\textbf{Yes}}, keeping equilibrium with atmosphere & \textcolor{SpringGreen3}{\textbf{Yes}}, barometer
            \\\hline
        \end{tabular}
        \caption{$T,P$ are the easiest quantities to measure and control, making $G(T,P)$ the easiest fundamental relationship to work with.}
    \end{table}
    \begin{align*}
        E(S,V)&\xrightarrow{\hspace{3.67cm}} dE= ~~ TdS-PdV % ~ for space
        \\
        F(T,V)&=E-TS \xrightarrow{\hspace{2cm}} dF= -SdT-PdV
        \\
        G(T,P)&=E-TS+pV \xrightarrow{\hspace{1.11cm}} dG= -SdT+VdP
        \\
        H(S,P)&=E+pV\xrightarrow{\hspace{2.04cm}} dH= ~~ TdS+VdP
    \end{align*}
    \newpage
\section*{Operations}
    The Derivative Crusher Algorithm is made of three operations and seven relations. 
    Looking at the derivatives of $G(T,P)=E-TS-PV$ more closely, certain constants can be defined: 
    \begin{align*}
        \pder{G}{T}[P]&=-S & \pder[2]{G}{T}[P]&=\coloredanswer{\pder{S}{T}[P]\equiv\frac{C_p}{T}}=\oo{T}\pder{Q}{T}[P]\tag{C1}\label{C1}
        \\
        \pder{G}{P}[T]&=V & \pder[2]{G}{P}[T]&=\coloredanswer{\pder{V}{P}[T]\equiv-\kappa V}\tag{C2}\label{C2}
    \end{align*}
    \begin{equation*}
        \qty[\pder{T}\pder{G}{P}[T]~]_{P}=\qty[\pder{P}\pder{G}{T}[P]~]_{T}=\coloredanswer{\pder{V}{T}[P]\equiv \alpha V}\tag{C3}\label{C3} % ~ is for space
    \end{equation*}
    $C_p$ is specific heat at constant pressure. 
    \\
    $\kappa$ is isothermal compressibility.
    \\
    $\alpha$ is volume coefficient of expansion.
    \\
    The derivative crusher algorithm is made of three partial derivative operations:
    \begin{enumerate}
        \item Bringing $y$ to the numerator:
        \begin{equation*}
            \coloredanswer{\pder{x}{y}[z]=\oo{\pder{y}{x}[z]}}\tag{D1}\label{D1}
        \end{equation*}
        \item Bringing $z$ to the numerator:
        \begin{equation*}
            \coloredanswer{\pder{x}{y}[z]=\frac{-\pder{z}{y}[x]}{\pder{z}{x}[y]}}\tag{D2}\label{D2}
        \end{equation*}
        \item Introduce new variable $w$:
        \begin{equation*}
            \coloredanswer{\pder{x}{y}[z]=\frac{\pder{x}{w}[z]}{\pder{y}{w}[z]}}\tag{D3}\label{D3}
        \end{equation*}
    \end{enumerate}
    The last crucial component are the Maxwell Relations between $S,V,T,P$:~\footnote{Note that in the one dimensional case, pressure is force (negative by convention): $P\rightarrow -F$ and $V\rightarrow X$}
    \begin{gather*}
        \coloredanswer{\pder{S}{V}[T]=\pder{P}{T}[V]}\tag{M1}\label{M1} % ~ is small space
        \\
        \coloredanswer{\pder{V}{S}[P]=\pder{T}{P}[S]}\tag{M2}\label{M2}
        \\
        \coloredanswer{\pder{S}{P}[T]=-\pder{V}{T}[P]}\tag{M3}\label{M3}
        \\
        \coloredanswer{\pder{P}{S}[V]=-\pder{T}{V}[S]}\tag{M4}\label{M4}
    \end{gather*}
    \newpage
\section*{Steps of the Algorithm}
    \begin{enumerate}
        \item Bring potentials ($E,F,G,H$) to the numerator (using \ref{D1} or \ref{D2}) and eliminate them using:
        \begin{align*}
            dE&=TdS-PdV
            &
            dF&=-SdT-PdV
            \\
            dG&=-SdT+VdP
            &
            dH&=TdS+VdP
        \end{align*}
        \item Bring $S$ to numerator using Maxwell Relations depending on what $z$ is in $\displaystyle\pder{x}{y}[z]$
        \begin{align*}
            z&=T\qquad\boxed{\pder{S}{V}[T]=\pder{P}{T}[V]\quad,\quad \pder{S}{P}[T]=-\pder{V}{T}[P]}
            \\
            z&=P,V\quad\boxed{\pder{S}{T}[P]=\frac{C_P}{T}\quad,\quad \pder{S}{T}[V]=\frac{C_V}{T}}
        \end{align*}
        \item Bring $V$ to numerator in order to get $\alpha$ or $\kappa$. 
        \begin{align*}
            \boxed{\frac{-\pder{V}{T}[P]}{\pder{V}{P}[T]}=\pder{P}{T}[V]=\frac{\alpha}{\kappa}}
        \end{align*}
        \item Eliminate $C_V$ in favor of $C_p,\alpha,\kappa$
        A trick for going from $C_V\rightarrow C_P$ uses another useful derivative trick:
        \begin{gather*}
            \pder{x}{y}[w]=\pder{x}{y}[z]+\pder{x}{z}[y]\pder{z}{x}[w]
            \\
            \pder{Q}{T}[V]=\pder{Q}{T}[P]+\pder{Q}{P}[T]\pder{P}{T}[V]
            \\
            T\pder{S}{T}[V]=T\pder{S}{T}[P]+T\pder{S}{P}[T]\pder{P}{T}[V]
            \\
            C_V=C_P+T\pder{S}{P}[T]\pder{P}{T}[V]
            \\
            \boxed{C_P-C_V=VT\frac{\alpha^2}{\kappa}}=R                 
        \end{gather*}
    \end{enumerate} 
\end{document}