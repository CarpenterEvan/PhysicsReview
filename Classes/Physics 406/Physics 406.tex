\documentclass{article}
\usepackage[utf8]{inputenc} % allows for non-ASCII characters
\usepackage[x11names]{xcolor}    % Color extensions
\usepackage[margin=1in]{geometry} % Formatting on page
\usepackage{array} % big arrays
\usepackage{amsmath} % Math symbols
\usepackage{amsthm}   
\usepackage{amssymb}
\usepackage[backend=bibtex]{biblatex} % bibliography
\usepackage{bm} % bold for greek letters
\usepackage{cancel}
\usepackage[format=hang]{caption}
\usepackage{enumitem} % Enumerating things
\usepackage{esint} % better double and triple Integrals
\usepackage{etoc}
\usepackage{fancyhdr} % Headers
    \pagestyle{fancy}
\usepackage{float}
\usepackage{gensymb}
\usepackage{physics} % easier commands for physics things
\usepackage{relsize}  % allows for larger/smaller math 
\usepackage{textcomp} % Gets rid of perthousand error
\usepackage{upquote}  % fixes quotes in verbatim environment
\usepackage{verbatim} % Allows for comment environment
\usepackage{tikz} % pictures
	\usetikzlibrary{calc}
	\usetikzlibrary{decorations.markings}
	\usetikzlibrary{3d}
	\usetikzlibrary{intersections}
\usepackage{pgfplots} % plots and graphics
	\pgfplotsset{compat=newest} % or use compat=1.6
	\usepgfplotslibrary{fillbetween}
\usepackage{graphicx} % plots and graphics
\usepackage{wrapfig}
\usepackage{xparse} % Better commands
\usepackage[hidelinks]{hyperref}% References--THIS GOES LAST

% Packages that this breaks without: amsmath, gensymb, physics, hyperref, xcolor, xparse, and possibly others
\numberwithin{equation}{section} % amsmath command that renews equation counter in each section\
\def\ints{\int_\mathcal{S}} % Surface integral
\def\intv{\int_\mathcal{V}} % Volume integral with V subscript
\def\intall{\int_{-\infty}^{\infty}} % Integral over all space
\def\iintall{\iint_{\rm All Space}} % Integral over all space
\def\iiintall{\iiint_{\rm All Space}} % Integral over all space
\def\ik{4\pi\epsilon_0} % Inverse k for EM
\def\lap{\mathcal{L}}
\def\answerline{ % double horizontal line placed 0.5 cm below text, space between lines is 0.07 cm, then 0.75 cm of space 
	\vspace{0.5 cm}
	\hrule
	\vspace{0.07 cm}
	\hrule
	\vspace{0.75 cm}\noindent} % don't indent text after the line
\NewDocumentCommand\length{O{3pt}}{\setlength\jot{#1}} % for align vertical spacing, there's probably a better way to do this locally
\NewDocumentCommand\ft{s O{n} O{L}}{ % I dont want to write \sin(stuff) \cos(stuff) every time in fourier transforms (ft)
    \IfBooleanTF{#1}{
        \sin(\frac{#2 \pi}{#3}x)
    }{
        \cos(\frac{#2 \pi}{#3}x)
    }
}

\NewDocumentCommand\dl{s}{\IfBooleanTF{#1}{}{\cdot} d\mathbf{l}} % quicker curve integral dl, if no star, it also makes the dot 
\NewDocumentCommand\da{s}{\IfBooleanTF{#1}{}{\cdot} d\mathbf{a}} % quicker surface integral da, if no star, it also makes the dot 
\NewDocumentCommand\oo{O{1} m} {\frac{#1}{#2}}   % reciprocal- 'one over', option to make it a regular fraction because oo is still quicker to type.
\NewDocumentCommand\thetitle{m O{}}{ \title{ \hypertarget{top}{\textbf{#1}} \\ \large {#2} } } %Bold title that is a hypertarget, optional bolded subtitle that's scaled properly 

\NewDocumentCommand \e {s m}{ % basis vector command
	\IfBooleanTF{#1} % \e{x} for x hat notation, \e*{x} for e_x notation
	{\mathbf{\hat{e}_{#2}}}
	{\bm{\hat{#2}}}} % from package bm, more powerful than \mathbf} 
\NewDocumentCommand \parametric {m}{ %command for getting boundary conditions with a "{" on the left
    \left\{ 
	\begin{gathered}
		\begin{matrix}
			#1
    		\end{matrix}
	\end{gathered}
	 \right.}
\NewDocumentCommand \der {s O{} m g}{ % custom derivative command
     \IfBooleanTF{#1} % if \der*, #1 is true, euler notation used, if \der , #1 is false, d/dx is used 
    {\IfNoValueTF{#4}  % g returns -NoValue- if no argument (read below)
        {\mathrm{D}_{#3}^{#2}}  % g is included so \der{x} and \der{f}{x} both have x in the denominator
        {\mathrm{D}_{#4}^{#2}#3}
            }
    {\IfNoValueTF{#4}
        {\frac{\mathrm{d}^{#2}}{\mathrm{d} #3^{#2}}}
        {\frac{\mathrm{d}^{#2} #3}{\mathrm{d} #4^{#2}}}
	}}    
\NewDocumentCommand \pder {s O{} m g d[]}{ % custom partial derivative command, allows for euler notation if starred, added d[] entry for parenthesis around derivative for quantities that are held constant 
	\IfBooleanTF{#1} % Star for euler notation
	   {\IfNoValueTF{#4} % Logic with #4 is so first argument gets put in denomenator if there is only one ( like \pder{x} ), but if there are two arguments (#4 is second argument) then the second argument is put in denominator \pder{f}{x}
		   {\partial_{#3}^{#2}}
		   {\partial_{#4}^{#2}#3}
			   }
	   {\IfNoValueTF{#5} % #5 is what is held constant
		   {\IfNoValueTF{#4}
			   {\frac{\partial^{#2}}{\partial #3^{#2}}}
			   {\frac{\partial^{#2} #3}{\partial #4^{#2}}}}
		   {{\IfNoValueTF{#4} 
			   {\left(\frac{\partial^{#2}}{\partial #3^{#2}}\right)_{\!#5}\!\!} % \! is thin negative space
			   {\left(\frac{\partial^{#2} #3}{\partial #4^{#2}}\right)_{\!#5}\!\!}}}
			   }} 
                
\NewDocumentCommand \header {m m m}{
	\fancyhead[L]{#1} 
	\fancyhead[C]{\hyperlink{top}{ \textbf{#2} }}
	\fancyhead[R]{#3}
	\fancyfoot[C]{--\thepage--}
	\pagestyle{fancy}
	\setlength\headheight{17pt}}
    
\NewDocumentCommand{\coloredanswer}{O{LavenderBlush2} m}{ % makes coloredbox with pretty color
	\mathchoice
	{\colorbox{#1}{$\displaystyle #2$}}
	{\colorbox{#1}{$\textstyle #2$}}
	{}
	{}}

\newcounter{probcount} % new counter for problem numbers, starts at 0 by default
\NewDocumentEnvironment{ problem } {O{} +b} %Problem environment for homework, autocounts numbers, behaves like a section and can get listed (and hyperlinked) in the toc. 
	{\addtocounter{probcount}{1} % increase counter at the beginning of every problem
	 \phantomsection % invisible page marker for hyperref
	 \addcontentsline{toc}{subsection}{Problem \theprobcount~{\it#1}} % add "Problem \theprobcount. \textit{#1}" to table of contents (toc) as if it were a subsection
	 \begin{trivlist} % begin unmarked ("trivial") list
	 \item {\bf Problem \theprobcount} {\it#1} % print Problem with the counter and optional text 
	 \item #2  % +b is for inputs that are long
	 \end{trivlist} % end list
	}{} % don't do anything then "problem" environment ends, irrelevant because trivlist is already ended
	
\def\UM{\colorbox{blue}{\textcolor{Gold1}{\textbf{University of Michigan}}}} % School Pride
\def\EC{\href{https://github.com/CarpenterEvan/PhysicsReview}{\(\mathbb{E}\textsc{van}~\mathbb{C}\textsc{arpenter}\)}}
\endinput
\thetitle{Statistical Mechanics}[\tt Physics 406 at \rm\small \UM]
\date{Winter Semester 2022}
\author{\EC}
    \def \pa {particle\ }
    \def \pas {particles\ }
    \def \ste {state\ }
    \def \micst {micro\ste}
    \def \macst {macro\ste}
    \def \dof {degree of freedom\ }
    \def \dofs {degrees of freedom\ }
\begin{document}
\maketitle
\invisiblelocaltableofcontents\label{toc:lecturetoc}
\begingroup
    \parindent 0cm \parfillskip 0cm \leftskip 0cm \rightskip 0cm
    \etocframedstyle[1]{}
    \etocsetstyle{subsection}
                  {}
                  {\leavevmode\leftskip 0cm\relax} {\normalsize\rm\etocname\nobreak\leaders\hbox{.}\hfill\tt p.\makebox[12 pt][r]{\etocpage} \par\noindent}
                  {\vspace{-0.25cm}}
    \tableofcontents\ref{toc:lecturetoc}
\endgroup
\newcounter{lecturecount} % starts at 0 by default
\NewDocumentEnvironment{ lecture } {O{} O{} +b} 
	{\addtocounter{lecturecount}{1} % increase counter
		\phantomsection % invisible marker for hyperref
		\addcontentsline{toc}{subsection}{\makebox[40 pt][l]{\texttt{\small #1 }}~\textbf{Lecture~\thelecturecount:}~\textsl{#2}} % add to table of contents
		\begin{trivlist} % begin unmarked ("trivial") list
		\item \colorbox{LavenderBlush2}{\textbf{Lecture \thelecturecount.}~(\texttt{#1})~ \textit{#2}} % Header
		\item #3  % +b is for inputs that are long
		\end{trivlist} % end list
	}{}
\newpage
    \begin{lecture}[Jan 05][States, Probability and Binomial Distribution]
        \begin{figure}[H]
            \centering
            \begin{tikzpicture}
                \draw[<->] (-2,0)--(2,0) node [anchor=west]{$x$};
                \draw[<->] (0,-2)--(0,2) node [anchor=south]{$p$};
                \draw[thick, black, fill] (1,1.5) circle(1pt);
            \end{tikzpicture}
            \caption{Phase space of 1-D \pa}
        \end{figure}
    \end{lecture}
    \begin{lecture}[Jan 10][Ensembles]
        Lagrange multipliers
        \begin{equation}
            S=-k\sum_rp_r\ln(p_r)
        \end{equation}
        Microcanonical ensemble: All accessible \micst are equally probable
    \end{lecture}
    \begin{lecture}[Jan 12][Finding total \micst]
        $N$ \pas in volume $V$ with energy between $E,E+\delta E$.
        Counting number of \micst by using phase space 
        \\
            simplifying example : a 1-D \pa has only $x$ and $p$. Plot in phase space
            Example in harmonic oscillator with ellipse and shading in phi(E) and Omega(E) Include text in caption explaining equations below it. 
        \\
        Moving to 3-D talk about \dofs and volume of $h_0$.
        \\
        Integrating to get Phi(E) with multiintegrals and then taylor approx to get Omega
        \\
        Quantum Description--> specify \micst with quantum numbers
        \\
        example with simp harmon oscill
    \end{lecture}
\begin{lecture}[Jan 19][More on Microcanonical Ensemble]
    $\Omega(E)=\#$ of states with energy between $E+\delta E$
    \\
    Describing energy levels of each particle, think N-cube
    \\
    Now \pa can interact!! Mechanical interactions and thermal interactions(both macro descriptions). 
    \\
    1 isolated system at equilibrium$\rightarrow$ same system but with partition, now 2 systems. $A^0$ is comprised of $A,A'$. Macro parameters of $A^0$ are for both states (N,V,E,T,\ldots).
    \\
    \begin{description}
        \item [Thermal Interaction] External parameters of $A,A'$ are fixed but mean energy transferred from one system to the other as a result of purely thermal interactions called heat. Probabilities of energy states can change when systems interact $P(r)$
        \item [Mechanical Interaction] External Parameters of $A,A'$ change, one does \emph{work} on the other! This causes the mean energies of $A,A'$ to change. 
    \end{description}
    $$\overline{E}=\sum_rp_rE_r$$
    \\
    Pure thermal and purely mech example in inf sqwell
\end{lecture}
\begin{lecture}[Jan 24]
    Pure thermal interaction changes $p_r$
    \\
    Pure mechanical interaction changes $E_r$
\end{lecture}
\end{document}