\documentclass{article}
\usepackage[utf8]{inputenc} % allows for non-ASCII characters
\usepackage[x11names]{xcolor}    % Color extensions
\usepackage[margin=1in]{geometry} % Formatting on page
\usepackage{array} % big arrays
\usepackage{amsmath} % Math symbols
\usepackage{amsthm}   
\usepackage{amssymb}
\usepackage[backend=bibtex]{biblatex} % bibliography
\usepackage{bm} % bold for greek letters
\usepackage{cancel}
\usepackage[format=hang]{caption}
\usepackage{enumitem} % Enumerating things
\usepackage{esint} % better double and triple Integrals
\usepackage{etoc}
\usepackage{fancyhdr} % Headers
    \pagestyle{fancy}
\usepackage{float}
\usepackage{gensymb}
\usepackage{physics} % easier commands for physics things
\usepackage{relsize}  % allows for larger/smaller math 
\usepackage{textcomp} % Gets rid of perthousand error
\usepackage{upquote}  % fixes quotes in verbatim environment
\usepackage{verbatim} % Allows for comment environment
\usepackage{tikz} % pictures
	\usetikzlibrary{calc}
	\usetikzlibrary{decorations.markings}
	\usetikzlibrary{3d}
	\usetikzlibrary{intersections}
\usepackage{pgfplots} % plots and graphics
	\pgfplotsset{compat=newest} % or use compat=1.6
	\usepgfplotslibrary{fillbetween}
\usepackage{graphicx} % plots and graphics
\usepackage{wrapfig}
\usepackage{xparse} % Better commands
\usepackage[hidelinks]{hyperref}% References--THIS GOES LAST

% Packages that this breaks without: amsmath, gensymb, physics, hyperref, xcolor, xparse, and possibly others
\numberwithin{equation}{section} % amsmath command that renews equation counter in each section\
\def\ints{\int_\mathcal{S}} % Surface integral
\def\intv{\int_\mathcal{V}} % Volume integral with V subscript
\def\intall{\int_{-\infty}^{\infty}} % Integral over all space
\def\iintall{\iint_{\rm All Space}} % Integral over all space
\def\iiintall{\iiint_{\rm All Space}} % Integral over all space
\def\ik{4\pi\epsilon_0} % Inverse k for EM
\def\lap{\mathcal{L}}
\def\answerline{ % double horizontal line placed 0.5 cm below text, space between lines is 0.07 cm, then 0.75 cm of space 
	\vspace{0.5 cm}
	\hrule
	\vspace{0.07 cm}
	\hrule
	\vspace{0.75 cm}\noindent} % don't indent text after the line
\NewDocumentCommand\length{O{3pt}}{\setlength\jot{#1}} % for align vertical spacing, there's probably a better way to do this locally
\NewDocumentCommand\ft{s O{n} O{L}}{ % I dont want to write \sin(stuff) \cos(stuff) every time in fourier transforms (ft)
    \IfBooleanTF{#1}{
        \sin(\frac{#2 \pi}{#3}x)
    }{
        \cos(\frac{#2 \pi}{#3}x)
    }
}

\NewDocumentCommand\dl{s}{\IfBooleanTF{#1}{}{\cdot} d\mathbf{l}} % quicker curve integral dl, if no star, it also makes the dot 
\NewDocumentCommand\da{s}{\IfBooleanTF{#1}{}{\cdot} d\mathbf{a}} % quicker surface integral da, if no star, it also makes the dot 
\NewDocumentCommand\oo{O{1} m} {\frac{#1}{#2}}   % reciprocal- 'one over', option to make it a regular fraction because oo is still quicker to type.
\NewDocumentCommand\thetitle{m O{}}{ \title{ \hypertarget{top}{\textbf{#1}} \\ \large {#2} } } %Bold title that is a hypertarget, optional bolded subtitle that's scaled properly 

\NewDocumentCommand \e {s m}{ % basis vector command
	\IfBooleanTF{#1} % \e{x} for x hat notation, \e*{x} for e_x notation
	{\mathbf{\hat{e}_{#2}}}
	{\bm{\hat{#2}}}} % from package bm, more powerful than \mathbf} 
\NewDocumentCommand \parametric {m}{ %command for getting boundary conditions with a "{" on the left
    \left\{ 
	\begin{gathered}
		\begin{matrix}
			#1
    		\end{matrix}
	\end{gathered}
	 \right.}
\NewDocumentCommand \der {s O{} m g}{ % custom derivative command
     \IfBooleanTF{#1} % if \der*, #1 is true, euler notation used, if \der , #1 is false, d/dx is used 
    {\IfNoValueTF{#4}  % g returns -NoValue- if no argument (read below)
        {\mathrm{D}_{#3}^{#2}}  % g is included so \der{x} and \der{f}{x} both have x in the denominator
        {\mathrm{D}_{#4}^{#2}#3}
            }
    {\IfNoValueTF{#4}
        {\frac{\mathrm{d}^{#2}}{\mathrm{d} #3^{#2}}}
        {\frac{\mathrm{d}^{#2} #3}{\mathrm{d} #4^{#2}}}
	}}    
\NewDocumentCommand \pder {s O{} m g}{ % custom partial derivative command, allows for euler notation if starred
     \IfBooleanTF{#1}
        {\IfNoValueTF{#4}
            {\partial_{#3}^{#2}}
            {\partial_{#4}^{#2}#3}
                }
        {\IfNoValueTF{#4}
            {\frac{\partial^{#2}}{\partial #3^{#2}}}
            {\frac{\partial^{#2} #3}{\partial #4^{#2}}}
                }} 
                
\NewDocumentCommand \header {m m m}{
	\fancyhead[L]{#1} 
	\fancyhead[C]{\hyperlink{top}{ \textbf{#2} }}
	\fancyhead[R]{#3}
	\fancyfoot[C]{--\thepage--}
	\pagestyle{fancy}
	\setlength\headheight{17pt}}
    
\NewDocumentCommand{\coloredanswer}{O{LavenderBlush2} m}{ % makes coloredbox with pretty color
	\mathchoice
	{\colorbox{#1}{$\displaystyle #2$}}
	{\colorbox{#1}{$\textstyle #2$}}
	{}
	{}}

\newcounter{probcount} % new counter for problem numbers, starts at 0 by default
\NewDocumentEnvironment{ problem } {O{} +b} %Problem environment for homework, autocounts numbers, behaves like a section and can get listed (and hyperlinked) in the toc. 
	{\addtocounter{probcount}{1} % increase counter at the beginning of every problem
	 \phantomsection % invisible page marker for hyperref
	 \addcontentsline{toc}{subsection}{Problem \theprobcount. {\it#1}} % add "Problem \theprobcount. \textit{#1}" to table of contents (toc) as if it were a subsection
	 \begin{trivlist} % begin unmarked ("trivial") list
	 \item {\bf Problem \theprobcount.} {\it#1} % print Problem with the counter and optional text 
	 \item #2  % +b is for inputs that are long
	 \end{trivlist} % end list
	}{} % don't do anything then "problem" environment ends, irrelevant because trivlist is already ended
	
\def\UM{\colorbox{blue}{\textcolor{Gold1}{\textbf{University of Michigan}}}} % School Pride
\def\EC{\href{https://github.com/CarpenterEvan/PhysicsReview}{\(\mathbb{E}\textsc{van}~\mathbb{C}\textsc{arpenter}\)}}
\endinput
\header{Physics 406}{}{Evan Carpenter}
\thetitle{Physics 406 Homework}
\date{Winter 2022}
\author{\EC}
\begin{document}
\maketitle
\section{Homework 1}
\begin{problem}
    problem 1
\end{problem}
\begin{problem}
    problem 2
\end{problem}
\begin{problem}
    problem 3
\end{problem}
\begin{problem}
    Suppose that a particle moving in one dimension is confined to $x>0$, and it's energy is $E=\frac{p^2}{2m}+mgx$ Make a sketch to indicate what
    region of classical phase space is accessible to this particle if its energy lies between $E_0$ and $E_0+\delta E_0$.
    \begin{figure}[H]
        \centering
        \begin{tikzpicture}
            \draw[<->] (-2,0)--(2,0) node [anchor=west]{$x$};
            \draw[<->] (0,-2)--(0,2) node [anchor=south]{$p$};
            \draw[scale=0.5, domain=-2:2, smooth, variable=\x, blue] plot({-\x*\x+1}, {\x});
            \draw[scale=0.5, domain=-2:2, smooth, variable=\x, red] plot({-\x*\x+2}, {\x});
        \end{tikzpicture}
        \caption{Particle constrained}
    \end{figure}
\end{problem}

\newpage
\section{Homework 2}
\setcounter{probcount}{0}

\begin{problem}
    \begin{enumerate}[label=(\alph*)]
        \item Show that the number of states $\phi(E)$ with energy less than $E$, for a particle of mass $m$ in a cubical box of side $L$ is:
        $$\phi(E)=\frac{\pi}{6}\qty(\frac{L}{\pi\hbar})^3\qty(2mE)^{3/2}$$
        Hint: Use the energy levels 2.1.3 in Reif and treat the n as continuous variables.
        \item Calculate $\Omega(E)$
        \item A nitrogen molecule at room temperature has a typical energy of $6\times10^{-14}$ergs. Calculate $\phi(E)$ for a particle in a box of side length 10cm. Also calculate $\Omega(E)$ assuming $\delta E=10^{-24}$ergs
    \end{enumerate}
    \answerline

\end{problem}\newpage
\begin{problem}[Reif 2.4]
    Consider an isolated system consisting of a large number $N$ of weakly interacting localized particles of spin $\frac{1}{2}$. Each particle has a magnetic moment $\mu$ which can point either parallel or antiparallel to an applied field $H$. The energy od the system is then $E=-(n_1-n_2)\mu H$, where $n_1$ is the number of spins aligned parallel to $H$ and $n_2$ is the number of spins aligned antiparallel to $H$. 
    \begin{enumerate}[label=(\alph*)]
        \item Consider the energy range between $E+\delta E$ where $\delta E$ is much smaller than $E$, but $E$ is still microscopically large, so $\mu H \ll \delta E\ll E$. What is $\Omega(E)$ (the total number of states  in the energy range)?
        \item Write down an expression for $\ln(\Omega(E))$ as a function of $E$. Simplify this expression by using Stirling's formula in it's simplest form: $$\ln(n!)\approx n\ln(n)-n$$
        \item Assume that the energy $E$ is in a region where $\Omega(E)$ is appreciable~$\rightarrow$~that it is not close to the extreme possible values $\pm N\mu H$ which it can assume. In this case apply a Gaussian approximation to part (a) to obtain a simple expression for $\Omega(E)$ as a function of $E$.
    \end{enumerate}
    \answerline
\end{problem}\newpage
\def\dbar{{\mathchar'26\mkern-12mu d}} % found online
\begin{problem}[Reif 2.5]
    Consider the infinitesimal quantity $$A(x,y)dx+B(x,y)dy\equiv\dbar F$$
    \begin{enumerate}[label=(\alph*)]
        \item Suppose $\dbar F$ is an exact differential so that $F=F(x,y)$. Show that $A,B$ must satisfy the condition: $$\pder{A}{y}=\pder{B}{x}$$
        \item If $\dbar F$ is an exact differential, show that the integral $\int \dbar F$ evaluated along any clsoed path on the $xy$ plane must vanish.
    \end{enumerate}
    \answerline
\end{problem}\newpage
\begin{problem}[Reif 2.7]
    \begin{enumerate}[label=(\alph*)]
        \item Consider a particle confined to a cubical box. The possible energy levels are given by $$E=\frac{(\hbar \pi)^2}{2m}\qty[\qty(\frac{n_x}{L_x})^2+ \qty(\frac{n_y}{L_y})^2+ \qty(\frac{n_z}{L_z})^2]$$ Show that the force exerted by the particle in this state on a wall perpendicular to the $x$ axis is given by $$F_x=-\pder{E}{L_x}$$ while the length $L_x$ is changed quasi-statically by an amount $d L_x$.
        \item Calculate explicitly the pressure on this wall. By averaging over all posible states, find an expression for the mean pressure on this wall (Hint: Exploit the property that $\overline{n_x^2}=\overline{n_y^2}=\overline{n_z^2}$ must be true by symmetry.) Show that the mean pressure can be simply expressed in terms of mean energy $\overline{E}$ of the particle and the volume $V=L_xL_yL_z$ of the box.  
    \end{enumerate}
    \answerline
\end{problem}
\end{document}